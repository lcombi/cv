%% start of file `template.tex'.
%% Copyright 2006-2013 Xavier Danaux (xdanaux@gmail.com).
%
% This work may be distributed and/or modified under the
% conditions of the LaTeX Project Public License version 1.3c,
% available at http://www.latex-project.org/lppl/.


\documentclass[11pt,a4paper,sans]{moderncv}        % possible options include font size ('10pt', '11pt' and '12pt'), paper size ('a4paper', 'letterpaper', 'a5paper', 'legalpaper', 'executivepaper' and 'landscape') and font family ('sans' and 'roman')

% moderncv themes
\moderncvstyle{classic}                             % style options are 'casual' (default), 'classic', 'oldstyle' and 'banking'
\moderncvcolor{purple}                               % color options 'blue' (default), 'orange', 'green', 'red', 'purple', 'grey' and 'black'
%\renewcommand{\familydefault}{\sfdefault}         % to set the default font; use '\sfdefault' for the default sans serif font, '\rmdefault' for the default roman one, or any tex font name
%\nopagenumbers{}                                  % uncomment to suppress automatic page numbering for CVs longer than one page

% character encoding
\usepackage[utf8]{inputenc}   
                    % if you are not using xelatex ou lualatex, replace by the encoding you are using
%\usepackage{CJKutf8}                              % if you need to use CJK to typeset your resume in Chinese, Japanese or Korean

% adjust the page margins
\usepackage[scale=0.75]{geometry}
%\setlength{\hintscolumnwidth}{3cm}                % if you want to change the width of the column with the dates
%\setlength{\makecvtitlenamewidth}{10cm}           % for the 'classic' style, if you want to force the width allocated to your name and avoid line breaks. be careful though, the length is normally calculated to avoid any overlap with your personal info; use this at your own typographical risks...

% personal data
\name{Luciano}{Combi}
\title{}                               % optional, remove / comment the line if not wanted
\address{D 80 n 930}{La Plata}{Argentina}% optional, remove / comment the line if not wanted; the "postcode city" and and "country" arguments can be omitted or provided empty
\phone[mobile]{+54~(0221)~6389~686}                   % optional, remove / comment the line if not wanted
\phone[fixed]{+54 (221)~425~4909 [102]}                    % optional, remove / comment the line if not wanted                      % optional, remove / comment the line if not wanted
\email{lcombi@iar.unlp.edu.ar}                               % optional, remove / comment the line if not wanted
                % optional, remove / comment the line if not wanted
                      % optional, remove / comment the line if not wanted; '64pt' is the height the picture must be resized to, 0.4pt is the thickness of the frame around it (put it to 0pt for no frame) and 'picture' is the name of the picture file
                                 % optional, remove / comment the line if not wanted

% to show numerical labels in the bibliography (default is to show no labels); only useful if you make citations in your resume
%\makeatletter
%\renewcommand*{\bibliographyitemlabel}{\@biblabel{\arabic{enumiv}}}
%\makeatother
%\renewcommand*{\bibliographyitemlabel}{[\arabic{enumiv}]}% CONSIDER REPLACING THE ABOVE BY THIS

% bibliography with mutiple entries
%\usepackage{multibib}
%\newcites{book,misc}{{Books},{Others}}
%----------------------------------------------------------------------------------
%            content
%----------------------------------------------------------------------------------
\begin{document}
%\begin{CJK*}{UTF8}{gbsn}                          % to typeset your resume in Chinese using CJK
%-----       resume       ---------------------------------------------------------
\makecvtitle

\section{Academic Formation}
\cventry{2011-2016}{Ms.S. in Physics}{Department of physics}{Faculty of Exact Sciences}{\textit{Universidad Nacional de La Plata} (UNLP)}{}
\cvitem{}{Average mark: 9.60/10}
\cvitem{}{Degree thesis:  between General Relativity and Teleparallel Gravity. Mark: 10/10. Supervisor: Gustavo E. Romero.}


\cventry{2017-Present}{Ph.D. in Physics}{Department of physics}{Faculty of Exact Sciences}{UNLP}{}
\cvitem{}{Supervisor: Gustavo E. Romero.}

\section{Current position}
\cventry{2017-Present}{Doctoral Fellow}{CONICET}{}{}{}  % arguments 3 to 6 can be left empty
\cvitem{}{Supervisor: Gustavo E. Romero.}
\cvitem{}{Research field: gravitation, General Relativity, cosmology, black holes, local physics and cosmology, astrophysics, foundations of physics, scientific philosophy.}
\cvitem{}{Place: Instituto Argentino de Radioastronomía, Bs.As., Argentina}

\section{Ph.D. Courses}
\cventry{2017}{Elements of Quantum Field Theory I}{}{}{}{}  % arguments 3 to 6 can be left empty
\cvitem{}{Department of Physics, Faculty of Exact Sciences, UNLP}
\cvitem{}{Professor: Horacio Falomir}
\cvitem{}{Duration: 1 semester}

\section{Fellowships}
\cventry{2017}{Ph.D. Research Fellow}{CONICET}{(National Research Council)}{}{}  % arguments 3 to 6 can be left empty
\cvitem{}{Supervisor: Gustavo E. Romero.}
\cvitem{}{Place: Instituto Argentino de Radioastronomía, Bs.As., Argentina.}


\section{Teaching Experience}

\cvitem{2015}{\textbf{Undergraduate Teaching Assistant} of Calculus II, Department of Mathematics, Faculty of Exact Sciences, UNLP. \textbf{Period}: 1th semester}

\cvitem{2015 - 2017}{\textbf{Undergraduate Teaching Assistant}, Department of Physics, Faculty of Exact Sciences, UNLP. Courses given: Linear Algebra, General Physics I, General Physics II}

\cvitem{2015 - 2017}{\textbf{Undergraduate Teaching Assistant} Faculty of Engineering, UNLP.  \textbf{Course}: Physics I (Laboratory duties)} 

\cvitem{2017- Present}{\textbf{Graduate Teaching Assistant} Department of Physics, Faculty of Exact Sciences, UNLP. Courses given: Gravitation, General Physics III, Methods in Mathematical Physics. \textbf{Current course}: Mechanics I}

\section{Workshops and Schools}

\cvitem{2016}{\textit{Journeys in theoretical physics}, (ICTP-Perimeter Institute). \textbf{Place:} Sao Paulo, Brasil. \textbf{Duration:} 1 week (40 hs). \textbf{Funding:} ICTP-SAIFR}


\cvitem{2016}{\textit{$f(R)$ theories of gravity}, (FCGALP, UNLP). \textbf{Place:} La Plata, Argentina. \textbf{Duration:} 1 week (40 hs).}


\cvitem{2018}{\textit{LAPIS: Cosmology in the era of large surveys}, (FCGALP, UNLP). \textbf{Place:} La Plata, Argentina. \textbf{Duration:} 1 week (40 hs).\textbf{Funding:} UNLP }


\cvitem{2018}{\textit{International Pulsar Timing Array, student week}, (NRAO). \textbf{Place:} New Mexico, USA. \textbf{Duration:} 1 week (40 hs). \textbf{Funding:} NANOGrav}


\cvitem{2018}{\textit{First Biennial Midwest Summer School in Philosophy of Physics}, (University of Chicago). \textbf{Place:} Chicago, USA. \textbf{Duration:} 1 week (40 hs). \textbf{Funding:} Templeton Foundation}


\cvitem{2018}{\textit{The Sound of Space-Time: The dawn of Gravitational Wave Science}, (ICTP-SAIFR). \textbf{Place:} Sao Paulo, Brasil.\textbf{Duration:} 3 weeks (120 hs). \textbf{Funding:} ICTP-SAIFR}


\cvitem{2019}{\textit{North American Einstein Toolkit Workshop}, (RIT). \textbf{Place:} Rochester, USA. \textbf{Duration:} 3 days. \textbf{Funding:} CCRG-RIT}

\section{Research stays abroad}

\cvitem{2018}{\textit{West Virginia University},
\textbf{Place:} Morgantown, West Virginia, USA.
\textbf{Duration} 1 mont
\textbf{Funding:} NANOGrav Collaboration}

\cvitem{2018}{\textit{Rochester Institute of Technology},
\textbf{Place:} Rochester, NY, USA.
\textbf{Duration} 6 months.
\textbf{Funding:} Center for Computational Relativity and Gravitation, RIT
\textbf{Project:} MHD simulations of spinning binary black hole systems.}



\section{Scientific meetings}

\subsection{Invited presentation}

\cvitem{2017}{\textit{PuMA proyect: Pulsar Monitoring in Argentina} (in Spanish)}
\cvitem{}{Encuentro de Estudiantes de Astronomía, Buenos Aires, Argentina. September 2017}

\cvitem{2018}{\textit{First Pulsar Observations in South America}}
\cvitem{}{Binational meeting SOCHIAS-AAA, La Serena, Chile. Octubre 16}


\subsection{Contribution presentation}

\cvitem{2015}{\textit{Inconsistency within the Everett interpretation of Quantum Mechanics}}
\cvitem{}{First Latinamerican congress of Scientific Philosophy (In honor to Mario Bunge), Buenos Aires, Argentina. October 2015}

\cvitem{2019}{\textit{Gravitational wave science and pulsars in Argentina}}
\cvitem{}{Grav19, Cordoba, Argentina. April 12}

\subsection{Posters}

\cvitem{2015}{\textit{Force between cylindric magnets: Theory and experiment} (in Spanish)}
\cvitem{}{Luciano Combi, Lucas Pili, Pablo Pisani, Fernando Monticelli}
\cvitem{}{100ª Anual Meeting of the Asociaci\'on Argentina de F\'isica (AFA), September 2015}

\cvitem{2017}{\textit{Intensive monitoring of pulsars in the south hemisphere} (in Spanish)}
\cvitem{}{Luciano Combi, Jorge Combi, Federico García, Guillermo Gancio, Carlos Lousto}
\cvitem{}{Anual Meeting of the Asocaci\'on Argentina de Astronom\'ia (AAA), September 2017}

\cvitem{2018}{\textit{Orbits in inhomogenous expanding space-times}}
\cvitem{}{Luciano Combi, Eduardo Guti\'errez}
\cvitem{}{LAPIS: Cosmology in the era of large surveys, April, 2018}

\cvitem{2018}{\textit{The IAR observatory and the PuMA project}}
\cvitem{}{Luciano Combi, Guillermo Gancio, Carlos Lousto}
\cvitem{}{IPTA international meeting, Albuquerque, USA}

\section{Outreach}

\cvitem{2018}{\textit{Friday talks in the Planetarium}: Gravitational waves and pulsars}
\cvitem{}{Planetarium, La Plata, Argentina}

\cvitem{2018}{\textbf{Outreach} department member of Argentine Institute of Radioastronomy}
\cvitem{}{In charge of social media management and guide for primary school and high-school visits to the Institute}

\section{Languages}
\cvitemwithcomment{Native}{Spanish}{}
\cvitemwithcomment{}{English}{}
\cvitemwithcomment{}{French}{}

\section{Programming}
\cvitem{}{Mathematica (advanced), Python, C}

\section{Memberships}

\cvitem{}{PuMA (\textit{Pulsar Monitoring in Argentina} collaboration.
\textbf{Status:} Full member. \textbf{Place:} Argentine Institute of Radioastronomy (IAR), La Plata, Argentina}

\section{Awards}


\cvitem{2017}{\textbf{Joaqu\'in V. Gonzales award} for distinguished graduate of the National University of La Plata. Given by the City Government of La Plata, Capital of Buenos Aires.}


\section{Other activities}

\cvitem{}{\textbf{Reviewer in scientific journals} and institutions:}
\cvitem{}{Astrophysics and Space Science (Springer)}
\cvitem{}{Gravitation and Cosmology (Springer)}
\cvitem{}{Estonian Research Council (ETIS)}


\pagebreak

\section{Publications}

\subsection{Refereed papers in international journals}

\cventry{1}{Inconsistency within the Everett interpretation of Quantum Mechanics}{}{}{}{}
\cvitem{}{Luciano Combi, Gustavo E. Romero.}
\cvitem{}{\textit{Methateoria} (ISSN 1853-2322) \textbf{7}, 47-53, 2017}

\cventry{2}{Gravitational energy and radiation of a charged black hole}{}{}{}{}
\cvitem{}{Luciano Combi, Gustavo E. Romero.}
\cvitem{}{\textit{Classical and  Quantum Gravity}, \textbf{34}, 195008, 2017}

\cventry{3}{Is Teleparallel Gravity really equivalent to General Relativity?}{}{}{}{}
\cvitem{}{Luciano Combi, Gustavo E. Romero.}
\cvitem{}{\textit{Annalen der Physik}, 1700175, 2018}

\cventry{4}{A note on geodesics in inhomogeneous expanding spacetimes}{}{}{}{}
\cvitem{}{D. Perez, G.E. Romero, Luciano Combi, E.M. Guti\'errez.}
\cvitem{}{\textit{Classical and  Quantum Gravity}, \textbf{36}, 055002, 2019}

\cventry{5}{Electromagnetic fields and charges in expanding universes}{}{}{}{}
\cvitem{}{Luciano Combi, Gustavo E. Romero}
\cvitem{}{\textit{Physical Review D}, \textbf{99}, 064017, 2019}

\cventry{6}{Upgraded antennas for pulsar observations in the Argentine Institute of Radio astronomy}{}{}{}{}
\cvitem{}{G. Gancio, C.O. Lousto, Luciano Combi, S. del Palacio, F. G. Lopez Armengol, J. A. Combi, F. García, P.Kornecki, A. L. Müller, E. Gutierrez, and F. Hauscarriaga}
\cvitem{}{\textit{Astronomy and Astrophysics}, 2019 (Submitted)}

\cventry{7}{Properties and reference frames in General Relativity}{}{}{}{}
\cvitem{}{Luciano Combi, Federico Lopez Armengol}
\cvitem{}{\textit{Philosophy of Science}, 2019 (Submitted)}

\cventry{8}{Is space-time material?}{}{}{}{}
\cvitem{}{Luciano Combi}
\cvitem{}{\textit{Materialism today}, Syntheses (Springer), 2019 (Forthcoming)}


\section{Observational experience}

\subsection{Telegrams}

\cventry{1}{Follow up of the radio flare from the magnetar XTE J1810-197 at 1.4 GHz}{}{}{}{}
\cvitem{}{Del Palacio, S.; Garcia, F.; Combi, L.; Lopez Armengol, F.; Gancio, G.; Muller, A. L.; Kornecki, P., on behalf of the PuMA Collaboration}
\cvitem{}{\textit{The Astronomer's Telegram}, \textbf{12323}, 2018}

\cventry{2}{Radio observations following the recent glitch of Vela Pulsar (PSR B0833-45)}{}{}{}{}
\cvitem{}{F. G. Lopez Armengol, C. O. Lousto, S. del Palacio , F. Garcia, L. Combi, J. A. Combi , G. Gancio , A. L. Mueller, P. Kornecki, on behalf of the PuMA Collaboration}
\cvitem{}{\textit{The Astronomer's Telegram}, \textbf{12482}, 2019}


\end{document}


%% end of file `template.tex'.
