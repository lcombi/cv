%% start of file `template.tex'.
%% Copyright 2006-2013 Xavier Danaux (xdanaux@gmail.com).
%
% This work may be distributed and/or modified under the
% conditions of the LaTeX Project Public License version 1.3c,
% available at http://www.latex-project.org/lppl/.


\documentclass[11pt,a4paper,sans]{moderncv}        % possible options include font size ('10pt', '11pt' and '12pt'), paper size ('a4paper', 'letterpaper', 'a5paper', 'legalpaper', 'executivepaper' and 'landscape') and font family ('sans' and 'roman')
\usepackage{soul}
% moderncv themes
\moderncvstyle{classic}                             % style options are 'casual' (default), 'classic', 'oldstyle' and 'banking'
\moderncvcolor{black}                               % color options 'blue' (default), 'orange', 'green', 'red', 'purple', 'grey' and 'black'
%\renewcommand{\familydefault}{\sfdefault}         % to set the default font; use '\sfdefault' for the default sans serif font, '\rmdefault' for the default roman one, or any tex font name
%\nopagenumbers{}                                  % uncomment to suppress automatic page numbering for CVs longer than one page

% character encoding
\usepackage[utf8]{inputenc}   
                    % if you are not using xelatex ou lualatex, replace by the encoding you are using
%\usepackage{CJKutf8}                              % if you need to use CJK to typeset your resume in Chinese, Japanese or Korean
% adjust the page margins
\usepackage[scale=0.75]{geometry}
\setlength{\hintscolumnwidth}{2cm}                % if you want to change the width of the column with the dates
\setlength{\makecvtitlenamewidth}{10cm}           % for the 'classic' style, if you want to force the width allocated to your name and avoid line breaks. be careful though, the length is normally calculated to avoid any overlap with your personal info; use this at your own typographical risks...

% My commands
\DeclareRobustCommand{\luc}[1]{ {\begingroup\color{purple}#1\endgroup} }
\newcommand{\lc}{{ \textcolor{purple}{Luciano Combi} }} 



% personal data
\name{Luciano}{Combi}

% to show numerical labels in the bibliography (default is to show no labels); only useful if you make citations in your resume
%\makeatletter
%\renewcommand*{\bibliographyitemlabel}{\@biblabel{\arabic{enumiv}}}
%\makeatother
%\renewcommand*{\bibliographyitemlabel}{[\arabic{enumiv}]}% CONSIDER REPLACING THE ABOVE BY THIS

% bibliography with mutiple entries
%\usepackage{multibib}
%\newcites{book,misc}{{Books},{Others}}
%----------------------------------------------------------------------------------
%            content
%----------------------------------------------------------------------------------
\begin{document}
%\begin{CJK*}{UTF8}{gbsn}                          % to typeset your resume in Chinese using CJK
%-----       resume       ---------------------------------------------------------
\makecvtitle

I am a relativistic astrophysicist interested in the violent phenomena that occur around strong gravitational fields. I use simulations and semi-analytical models to study the electromagnetic radiation from compact objects such as black holes, supermassive black holes binaries, neutron star mergers, and other systems. I also work with observations of radio pulsars and radio-transients.
 

\section{Personal information}
\cvitem{Adress}{Instituto Argentino de Radioastronomía, Buenos Aires, Argentina}
\cvitem{Citizenship}{Argentina}
\cvitem{Webpage}{\url{sites.google.com/view/lucianocombi}}
\cvitem{Email}{\url{lcombi@iar.unlp.edu.ar}}

\section{Education}
\cventry{2011 - 2016}{Master in Physics}{Department of physics}{Faculty of Exact Sciences}{\textit{Universidad Nacional de La Plata} (UNLP)}{}
\cvitem{}{Average mark: 9.60/10}
\cvitem{}{Degree thesis:  Equivalence between General Relativity and Teleparallel Gravity. Mark: 10/10. Supervisor: Gustavo E. Romero.}

\cventry{2017 - 2022}{Ph.D. in Physics}{Department of physics}{Faculty of Exact Sciences}{UNLP}{}
\cvitem{}{Supervisor: Gustavo E. Romero.}
\cvitem{}{Degree thesis: Local effects of the cosmic expansion}

\section{Current position}
\cventry{2017 -}{CONICET Ph.D Fellow}{}{}{}{}  % arguments 3 to 6 can be left empty
\cvitem{}{Supervisor: Gustavo E. Romero.}
\cvitem{}{Place: Instituto Argentino de Radioastronom\'ia}

\section{Awards}

\cvitem{2017}{\textbf{Joaqu\'in V. Gonzales award} for distinguished graduate of the National University of La Plata. Given by the City Government of La Plata, Capital of Buenos Aires.}

\cventry{2017}{CONICET Fellowship}{}{5 year fellowship awarded by the National Research Council of Argentina}{}{}  % arguments 3 to 6 can be left empty

\cventry{2021}{Visiting Fellowship from Perimeter Institute}{}{Awarded one semester visiting fellowship at Perimeter Institute (full funding) to work with Dr. Daniel Siegel on binary neutron star mergers \textit{(Postponed: work started online due to COVID19)}}{}{}  % arguments 3 to 6 can be left empty

\cventry{2021}{AARMS award}{}{Third place for Best Graduate student talk in the Canadian Student and Postdoc Conference on Gravity}{}{}  % arguments 3 to 6 can be left empty

\section{Research stays abroad}

\cvitem{2018}{\textit{West Virginia University},
\textbf{Place:} Morgantown, West Virginia, USA.
\textbf{Duration} 1 month,
\textbf{Funding:} NANOGrav Collaboration,
\textbf{Project:} Timing of milisecond pulsar J0437-4715, with Michael Lam and Maura McLauhglin}

\cvitem{2019}{\textit{Rochester Institute of Technology},
\textbf{Place:} Rochester, NY, USA,
\textbf{Duration} 6 months,
\textbf{Funding:} Center for Computational Relativity and Gravitation, RIT,
\textbf{Project:} MHD simulations of spinning binary black hole systems, with Manuela Campanelli.}

%\cvitem{2021}{\textit{Perimeter Institute},
%\textbf{Place:} Waterloo, Canada,
%\textbf{Duration} 4 months,
%\textbf{Funding:} Visiting Fellowhship, PI,
%\textbf{Project:} Modelling fast ejecta emission from binary neutron star mergers, with Daniel Siegel.}

\section{Computational expertise}
\cvitem{Languages}{Mathematica, Python, C/C++, BASH, Jupyter}
\cvitem{HPC}{MPI/OMP, Einstein Toolkits (Cactus), GRMHD codes such as HARM3D and GRHydro}
\cvitem{Use of clusters}{ Frontera (TX, USA), BlueWaters (IL, USA), Niagara (ON, CAN)}

\section{Observational experience}
\cvitem{}{Radio observations of \textit{pulsars} with single dish Antennas at the Argentine Institute of Radioastronomy. Reduction and analysis of data. Software usage: \texttt{PRESTO},\texttt{PSRCHIVE},\texttt{Enterprise},\texttt{TEMPO2}}

\subsection{Telegrams}

\cventry{1}{Follow up of the radio flare from the magnetar XTE J1810-197 at 1.4 GHz}{}{}{}{}
\cvitem{}{Del Palacio, S.; Garcia, F.; Combi, L.; Lopez Armengol, F.; Gancio, G.; Muller, A. L.; Kornecki, P., on behalf of the PuMA Collaboration}
\cvitem{}{\textit{The Astronomer's Telegram}, \textbf{12323}, 2018}

\cventry{2}{Radio observations following the recent glitch of Vela Pulsar (PSR B0833-45)}{}{}{}{}
\cvitem{}{F. G. Lopez Armengol, C. O. Lousto, S. del Palacio , F. Garcia, L. Combi, J. A. Combi , G. Gancio , A. L. Mueller, P. Kornecki, on behalf of the PuMA Collaboration}
\cvitem{}{\textit{The Astronomer's Telegram}, \textbf{12482}, 2019}

\section{Teaching and mentoring experience}

\subsection*{Course assistant}
\cvitem{2015}{\textbf{Undergraduate teaching assistant} of Calculus II, Department of Mathematics, Faculty of Exact Sciences, UNLP. \textbf{Period}: 1th semester}

\cvitem{2015 - 2017}{\textbf{Undergraduate teaching assistant}, Department of Physics, Faculty of Exact Sciences, UNLP. Courses given: Linear Algebra, General Physics I, General Physics II}

\cvitem{2015 - 2017}{\textbf{Undergraduate teaching assistant} Faculty of Engineering, UNLP.  \textbf{Course}: Physics I (Laboratory duties)} 

\cvitem{2017 - 2019}{\textbf{Graduate teaching assistant} Department of Physics, Faculty of Exact Sciences, UNLP. Courses given: Gravitation, General Physics III, Methods in Mathematical Physics. Mechanics I}

\subsection*{Mentorship}

\cvitem{2019 - 2020}{\textbf{Thesis co-advisor} for the master's degree (\textit{Licenciatura}) in Astronomy, Valentina Sosa Fiscella.  \textbf{Topic}: High-precision timing of pulsar J0437-4715 from IAR} 

\section{Grants and funding}

\cvitem{2016}{Full funding given by the ICTP-Perimeter Institute for one-week school 'Journeys in theoretical physics' at ICTP, Sao Pablo, Brasil}

\cvitem{2018}{Partial funding given by NANOgrav for one month research visit at the West Virginia University, Morgantown, USA}

\cvitem{2018}{Partial funding given by the Templeton foundation for one-week school 'First Biennial Midwest Summer School in Philosophy of Physics' at University of Chicago, Chicago, USA}

\cvitem{2018}{Full funding given by the ICTP for three-week school 'The Sound of Spacetime' at ICTP, Sao Pablo, Brasil}

\cvitem{2019}{Full funding given by the CCRG for six month research visit at the Rochester Institute of Technology (Rochester, USA) (PI: Manuela Campanelli)}

\cvitem{2020}{Full funding given by the Perimeter Institute for four month research visit at the Perimeter Institute (Waterloo, Canada) (PI: Daniel Siegel)}


\cvitem{2020}{Collaborator in NSF grant: "MRI: Acquisition of a Computing System for Large Simulation Data Sets in Multimessenger Astrophysics” (PI: Manuela Campanelli)}

\cvitem{2021}{Collaborator in NSF grant: “Collaborative Research: Supermassive Binary Black Hole Mergers: Accretion Dynamics and Electromagnetic Output“ (at NSF Windows on the Universe: The Era of Multi-messenger Astrophysics) (PI: Manuela Campanelli and Julian Krolik)}


\section{Workshops and Schools}

\cvitem{2016}{\textit{Journeys in theoretical physics}, (ICTP-Perimeter Institute). \textbf{Place:} Sao Paulo, Brasil. \textbf{Duration:} 1 week (40 hs). \textbf{Funding:} ICTP-SAIFR}

\cvitem{2016}{\textit{$f(R)$ theories of gravity}, (FCGALP, UNLP). \textbf{Place:} La Plata, Argentina. \textbf{Duration:} 1 week (40 hs).}

\cvitem{2018}{\textit{LAPIS: Cosmology in the era of large surveys}, (FCGALP, UNLP). \textbf{Place:} La Plata, Argentina. \textbf{Duration:} 1 week (40 hs).\textbf{Funding:} UNLP }

\cvitem{2018}{\textit{International Pulsar Timing Array, student week}, (NRAO). \textbf{Place:} New Mexico, USA. \textbf{Duration:} 1 week (40 hs).}

\cvitem{2018}{\textit{First Biennial Midwest Summer School in Philosophy of Physics}, (University of Chicago). \textbf{Place:} Chicago, USA. \textbf{Duration:} 1 week (40 hs).}

\cvitem{2018}{\textit{The Sound of Space-Time: The dawn of Gravitational Wave Science}, (ICTP-SAIFR). \textbf{Place:} Sao Paulo, Brasil.\textbf{Duration:} 3 weeks (120 hs). \textbf{Funding:} ICTP-SAIFR}

\cvitem{2019}{\textit{North American Einstein Toolkit Workshop}, (RIT). \textbf{Place:} Rochester, USA. \textbf{Duration:} 3 days. \textbf{Funding:} CCRG-RIT}

\cvitem{2020}{\textit{TCAN on Binary Neutron Stars}, (RIT). \textbf{Place:} Rochester, USA. \textbf{Duration:} 5 days.}


\section{Scientific meetings}

\subsection{\textbf{Invited presentation}}

\cvitem{2017}{\textit{The PuMA project: Pulsar Monitoring in Argentina} (in Spanish)}
\cvitem{}{Encuentro de Estudiantes de Astronomía, Buenos Aires, Argentina. September 2017}

\cvitem{2018}{\textit{First Pulsar Observations in South America}}
\cvitem{}{Binational meeting SOCHIAS-AAA, La Serena, Chile. Octubre 16}


\subsection{\textbf{Contribution presentation}}

\cvitem{2015}{\textit{Inconsistency within the Everett interpretation of Quantum Mechanics}}
\cvitem{}{First Latin-American congress of Scientific Philosophy (In honor to Mario Bunge), Buenos Aires, Argentina. October 2015}

\cvitem{2019}{\textit{Gravitational wave science and pulsars in Argentina}}
\cvitem{}{Grav19, Cordoba, Argentina. April 12}

\cvitem{2019}{\textit{Dual jets in supermassive black hole binaries}}
\cvitem{}{Argentine Astronomical Association, Rosario, Argentina. October 13}

\cvitem{2021}{\textit{GRMHD simulations of binary neutron stars with weak interactions}}
\cvitem{}{Canadian Student and Postdoc Conference on Gravity, Memorial University of Newfoundland, Canada. May 4}

\cvitem{2021}{\textit{Accretion onto spinning supermassive black hole binaries}}
\cvitem{}{LISA Astrophysics Working Group Meeting, Institute of Computational Science (ICS), University of Zurich}

\

\subsection{\textbf{Posters and proceedings}}

\cvitem{2015}{\textit{Force between cylindric magnets: Theory and experiment} (in Spanish)}
\cvitem{}{Luciano Combi, Lucas Pili, Pablo Pisani, Fernando Monticelli}
\cvitem{}{100ª Anual Meeting of the Asociaci\'on Argentina de F\'isica (AFA), September 2015}

\cvitem{2017}{\textit{Intensive monitoring of pulsars in the south hemisphere} (in Spanish)}
\cvitem{}{Luciano Combi, Jorge Combi, Federico García, Guillermo Gancio, Carlos Lousto}
\cvitem{}{Anual Meeting of the Asocaci\'on Argentina de Astronom\'ia (AAA), September 2017}

\cvitem{2018}{\textit{Orbits in inhomogeneous expanding space-times}}
\cvitem{}{Luciano Combi, Eduardo Guti\'errez}
\cvitem{}{LAPIS: Cosmology in the era of large surveys, April, 2018}

\cvitem{2018}{\textit{The IAR observatory and the PuMA project}}
\cvitem{}{Luciano Combi, Guillermo Gancio, Carlos Lousto}
\cvitem{}{IPTA international meeting, Albuquerque, USA}

\cvitem{2020}{\textit{Developing a digital receiver for pulsar observations}}
\cvitem{}{Gancio, G., Lousto, C., Combi, L., García, F., and Colaboración PuMA}
\cvitem{}{Bolet\'in de la Asociaci\'on Argentina de Astronom\'ia, La Plata, Argentina, vol. 61, pp. 222–224, 2020}

\section{Outreach \& media}

\cvitem{2018}{\textit{Friday talks in the Planetarium}: Gravitational waves and pulsars. Outreach talk at Planetarium, La Plata, Argentina}

\cvitem{2018}{Member of the \textbf{outreach} department at Argentine Institute of Radioastronomy. In charge of social media management and guide for primary school and high-school visits to the Institute}

\cvitem{2019}{\href{https://www.iar.unlp.edu.ar/boletin/la-existencia-de-los-agujeros-negros/}{On the existence of black holes}, outreach article in the bi-monthly Radioastronomy Bulletin (spanish)}

\cvitem{2019}{\href{https://youtu.be/ZMbq3JNkc_U}{'A vision of the Argentine Institute of Radioastronomy'}, producer of the mini-documentary directed by Luciana Demichelis}

\cvitem{2020}{\href{https://www.iar.unlp.edu.ar/boletin/agujeros-de-gusano-y-otras-especulaciones/}{Wormholes and other speculations}, opinion column in the bi-monthly Radioastronomy Bulletin (spanish)}

\cvitem{2020}{\href{https://www.telam.com.ar/notas/202012/539298-cientificos-del-conicet-aportan-novedades-sobre-pulsares-que-solo-se-ven-en-el-hemisferio-sur.html}{Pulsar hunters}, media cover in CONICET and the Argentine National News Agency (spanish)}

\section{Languages}
\cvitemwithcomment{Native}{Spanish}{}
\cvitemwithcomment{Proficient}{English}{}
\cvitemwithcomment{Intermediate}{French}{}

\section{Memberships}
\cvitem{}{\href{http://puma.iar.unlp.edu.ar/}{PuMA (IAR)} (\textit{Pulsar Monitoring in Argentina} collaboration.
\textbf{Status:} Full member. \textbf{Place:} Argentine Institute of Radioastronomy (IAR), La Plata, Argentina}

\cvitem{}{\href{https://compact-binaries.org/}{Compact binaries (RIT)}. Research collaboration for multi-messenger astrophysics.}

\cvitem{}{\href{https://garra.iar.unlp.edu.ar/}{GARRA (IAR-FCGALP)} (\textit{Grupo de Astrofísica relativista y radioastronomía)}}

\cvitem{}{\href{https://pi-uog-relastro.gitlab.io/group_website/}{RelAstro (PI-U.Guelph)}  (\textit{Relativistic Astrophysics Group at Perimeter Institute and U. of Guelph})}


\section{Other activities}

\cvitem{}{\textbf{Reviewer in scientific journals} and institutions:}
\cvitem{}{Astrophysics and Space Science (Springer)}
\cvitem{}{Gravitation and Cosmology (Springer)}
\cvitem{}{Estonian Research Council (ETIS)}

\pagebreak

\section{Publications}

\subsection{Papers in major peer-reviewed journals}
\cventry{12}{GRMHD simulations of binary neutron star mergers with weak interactions I}{}{}{}{}
\cvitem{}{\lc, Daniel Siegel}
\cvitem{}{\textit{Astrophysical Journal}, (In prep.), (2021)}

%\cventry{2021}{Electromagnetic emission in spinning supermassive black hole binaries approaching merger}{}{}{}{}
%\cvitem{}{Eduardo Gutierrez, Luciano Combi, Scott Noble, Manuela Campanelli, }
%\cvitem{}{\textit{Astrophysical Journal}, (In prep.)}

%\cventry{2021}{Non-thermal flares in supermassive binary black holes}{}{}{}{}
%\cvitem{}{Eduardo Gutierrez, Luciano Combi, Gustavo Romero, Manuela Campanelli}
%\cvitem{}{\textit{Astrophysical Journal}, (In prep.)}

\cventry{11}{Accretion onto spinning black hole binaries: mini-disks structure and outflows}{}{}{}{}
\cvitem{}{\lc, F.G. Lopez Armengol, Manuela Campanelli, Scott Noble, Mark Avara, Julian Krolik,  Dennis Bowen}
\cvitem{}{\textit{Astrophysical Journal}, (In prep.), (2021)}

\cventry{10}{A superposed metric for spinning black hole binaries}{}{}{}{}
\cvitem{}{\lc, F.G. Lopez Armengol, Manuela Campanelli, Brennan Ireland, Scott Noble, Hiroyuki Nakano, Dennis Bowen}
\cvitem{}{\textit{Physical Review D}, 2021, (Accepted), (2021)}

\cventry{9}{Circumbinary Disk Accretion into Spinning Black Hole Binaries}{}{}{}{}
\cvitem{}{F.G. Lopez Armengol, \lc, Manuela Campanelli, Scott Noble, Dennis Bowen, Mark Avara}
\cvitem{}{\textit{Astrophysical Journal}, \textbf{913} 16, (2021)}

\cventry{8}{PSR J0437-4715: The Argentine Institute of Radioastronomy 2019-2020 Observational Campaign}{}{}{}{}
\cvitem{}{V. Sosa Fiscella, S. del Palacio, \lc, C.O. Lousto, F. G. Lopez Armengol, J. A. Combi, F. García, P.Kornecki, A. L. Müller, E. Gutierrez, and F. Hauscarriaga}
\cvitem{}{\textit{Astrophysical Journal}, \textbf{913} 158, (2021)}


\cventry{7}{Relativistic rigid systems and the cosmic expansion}{}{}{}{}
\cvitem{}{\lc, Gustavo E. Romero}
\cvitem{}{\textit{General Relativity and Gravitation}, 52:93, (2020)}

\cventry{6}{Upgraded antennas for pulsar observations in the Argentine Institute of Radio astronomy}{}{}{}{}
\cvitem{}{G. Gancio, C.O. Lousto, \lc, S. del Palacio, F. G. Lopez Armengol, J. A. Combi, F. García, P.Kornecki, A. L. Müller, E. Gutierrez, and F. Hauscarriaga}
\cvitem{}{\textit{Astronomy and Astrophysics, 633, A84 }, (2020)}


\cventry{5}{Electromagnetic fields and charges in expanding universes}{}{}{}{}
\cvitem{}{\lc, Gustavo E. Romero}
\cvitem{}{\textit{Physical Review D}, \textbf{99}, 064017, (2019)}

\cventry{4}{A note on geodesics in inhomogeneous expanding spacetimes}{}{}{}{}
\cvitem{}{D. Perez, G.E. Romero, \lc, E.M. Guti\'errez.}
\cvitem{}{\textit{Classical and  Quantum Gravity}, \textbf{36}, 055002, (2019)}

\cventry{3}{Is Teleparallel Gravity really equivalent to General Relativity?}{}{}{}{}
\cvitem{}{\lc, Gustavo E. Romero.}
\cvitem{}{\textit{Annalen der Physik}, 1700175, (2018)}

\cventry{2}{Gravitational energy and radiation of a charged black hole}{}{}{}{}
\cvitem{}{\lc, Gustavo E. Romero.}
\cvitem{}{\textit{Classical and  Quantum Gravity}, \textbf{34}, 195008, (2017)}

\cventry{1}{Inconsistency within the Everett interpretation of Quantum Mechanics}{}{}{}{}
\cvitem{}{\lc, Gustavo E. Romero.}
\cvitem{}{\textit{Methateoria} (ISSN 1853-2322) \textbf{7}, 47-53, (2017)}

\

\subsection{Chapter in books}

\cventry{2021}{Is space-time material?}{}{}{}{}
\cvitem{}{\lc}
\cvitem{}{\textit{Ontological and Epistemological Issues in Contemporary Materialism}, Syntheses-Springer, 2020 (Forthcoming)}


\end{document}


%% end of file `template.tex'.
